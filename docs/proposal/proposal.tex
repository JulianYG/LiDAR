\documentclass[dvips,12pt]{article}
\usepackage[pdftex]{graphicx}
\usepackage{url}
\setlength{\oddsidemargin}{0.25in}
\setlength{\textwidth}{6.5in}
\setlength{\topmargin}{0in}
\setlength{\textheight}{8.5in}

\begin{document}
\title{Car Detection Using RGB Image Geometry and Semantic Estimations}
\author{Yuanfang Wang, Yinghao Xu, Julian Gao}
\date{\today}
\maketitle

\section{Introduction}
%\begin{figure}
%\begin{center}
%\resizebox{6in}{!}{\includegraphics*{m42.jpg}}
%\end{center}
%
%\caption{The Orion Nebula, M42, recorded with the CDK20N telescope on the night
%of November 1, 2011. This is a composite of three 100-second photometric images
%in the Sloan g' (shown as blue), r' (shown as red), and i' (shown as green)
%bands. The intensities are displayed with logarithmic compression. Click to see
%the inner regions with square root compression. Highly reddened stars are
%brighter in the infrared (i') and appear slightly green in these images.
%\label{m42}}
%
%\end{figure}

\begin{itemize}
\item Not fainter than magnitude 19 (18 is better)
\item Not larger than $0.5^\circ$, but see below
\item Above the horizon at either observatory for several hours this fall
\end{itemize}

A single 100 second exposure with the 0.5 meter telescopes will reach magnitude
18 on a clear night.  Accurate quantitative measurements require a little
brighter, or longer total accumulated exposures.  The telescopes resolve 1
arcsecond in two pixels and have a field of view of $0.6^\circ$.  Larger fields
must be mosaics of several exposures. These factors will affect your
choices.  

For planetary imaging the CDK20's can take exposures as short as 0.01 seconds. 
The longest practical single exposure is about 300 seconds, but typically we
take 100 second exposures and add them in order to make small guiding
corrections between exposures. Use AstroCC with Stellarium to assure that the
targets are observable this season.

\section{Objectives}

Assuming that the best telescope for your work is one of the two 0.5 meters
(CDK20N at Moore Observatory, CDK20S at Mt. Kent), you will have a choice of
filters:  Sloan filter set (g, r, i, or z),  Johnson-Cousins (U, B, V, R, or I),
color imaging (B, G, R, or clear), and narrow band (S $[II]$, red continuum,
H$\alpha$,  O$[III]$.  Identify which filters are of interest.

A typical exposure time for a magnitude 12 star to about half saturation is 100
seconds, but it depends on the filter choice.  Based on this, estimate how many
exposures you will need, and what total time you require.  In some cases, for
example studying an eclipsing or variable star, or an exoplanet transit, you
would use only one filter and make many measurements over a night.  In others,
you may make only a few exposures in each filter, and try many different
filters.   Changing filter sets takes an operator and several minutes, but
changing filters within one set (e.g. a different Sloan filter) takes only a few
seconds.

We have other telescopes that may be available at Moore Observatory this season.
There is a wide field astrograph that has a field of view of $4^\circ$ and is a
fast $f/4$,  especially good for large nebula, comets, or surveys.  A 14-inch
(0.36 meter) Celestron  telescope can be equipped with a fast camera for
planetary imaging.  A 27-inch (0.7 meter)  corrected Dall-Kirkham is scheduled
to be be delivered to Australia this fall, although we are unsure of the actual
date it could see light yet.  

\section{Potential Problems \& Approaches}
 
\section{Experimental Methods \& Results}

\begin{thebibliography}{99}

\bibitem{gonzalez2012} Jonay I. Gonz\'{a}lez Hern\'{a}ndez, 
Pilar Ruiz-Lapuente,	
Hugo M. Tabernero,	
David Montes,	
Ramon Canal,	
Javier M\'{e}ndez	
and Luigi R. Bedin,
{No surviving evolved companions of the progenitor of SN1006},
Nature, {\bf 489}, 533-536 (2012).

\end{thebibliography}



\end{document}