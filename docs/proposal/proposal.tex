\documentclass[dvips,12pt]{article}
\usepackage[pdftex]{graphicx}
\usepackage{url}
\setlength{\oddsidemargin}{0.25in}
\setlength{\textwidth}{6.5in}
\setlength{\topmargin}{0in}
\setlength{\textheight}{8.5in}

\begin{document}
\title{Car Detection Using RGB Image Geometry and Semantic Estimations}
\author{Yuanfang Wang, Yinghao Xu, Julian Gao}
\date{\today}
\maketitle

\section{Introduction}
%\begin{figure}
%\begin{center}
%\resizebox{6in}{!}{\includegraphics*{m42.jpg}}
%\end{center}
%
%\caption{The Orion Nebula, M42, recorded with the CDK20N telescope on the night
%of November 1, 2011. This is a composite of three 100-second photometric images
%in the Sloan g' (shown as blue), r' (shown as red), and i' (shown as green)
%bands. The intensities are displayed with logarithmic compression. Click to see
%the inner regions with square root compression. Highly reddened stars are
%brighter in the infrared (i') and appear slightly green in these images.
%\label{m42}}
%
%\end{figure}
Car detection has long been a popular topic in computer vision field. With the rise of industrial attention in autonomous driving and research focus on convolutional neural networks (CNN), car detection has seen rapid development recently. Early car detection in autonomous driving relies heavily on expensive devices, such as LIDAR, to sample depth and norm information. Recent works have tried to perform car detection based simply on camera captured images, and have reached considerably high accuracy on specialized datasets such as KITTI. Famous ones include fast R-CNN\cite{}, RPN\cite{}, etc. However those methods only make use of image data, and subject to problems such as scale variation, occlusion, and truncation\cite{}. To overcome these deficiencies and achieve better accuracy, we propose a new method to incorporate LIDAR into on-board detection system, which shifts the costly part to offline. We want to use CNN to train an image-LIDAR model, that takes in an RGB image and outputs depth, norm, and semantic segmentation obtained from its LIDAR map. We then perform car detection as well as simple 3D reconstruction on these outputs.

\section{Objectives}

\section{Potential Problems \& Approaches}
 
\section{Experimental Methods \& Results}

\begin{thebibliography}{99}

\bibitem{gonzalez2012} Jonay I. Gonz\'{a}lez Hern\'{a}ndez, 
Pilar Ruiz-Lapuente,	
Hugo M. Tabernero,	
David Montes,	
Ramon Canal,	
Javier M\'{e}ndez	
and Luigi R. Bedin,
{No surviving evolved companions of the progenitor of SN1006},
Nature, {\bf 489}, 533-536 (2012).

\end{thebibliography}



\end{document}